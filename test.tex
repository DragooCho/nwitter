git add .
git commit -m "#4.3 File URL and Nweet"
git push origin master



git add .
git commit -m "#5 CONCLUSIONS end"
git push -u origin main

rm -rf 지울폴더
지정한 폴더 묻지도 따지지도 않고 전부 지우기

git clone https://github.com/DragooCho/nwitter.git



git fetch
패치하면 원격 저장소의 최신 업데이트 내용을 로컬에 가져올 수 있다.
가져온 변경사항을 현재 브랜치와 머지하려면 pull 명령어를 사용한다.

git pull origin master
원격 저장소의 변경 내용을 가져와 로컬 저장소에 반영된다. 패스트 포워드가 가능하면 바로 반영하고 그렇지 않으면 머지 커밋을 추가로 만든다.

내가 작업한 내용을 깨끗히 유지하고 싶으면 머지 커밋을 만들지 않고 리베이스로 변경사항을 반영할 수도 있다.

git pull origin maseter --rebase
이렇게 하면 원격저장소에 있는 변경사항을 먼저 쌓고 그 위에 내가 작업한 변경사항을 쌓는 방식이다.